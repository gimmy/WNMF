\documentclass[a4paper]{scrartcl}
\usepackage[italian]{babel} 
\usepackage[utf8]{inputenc}
\usepackage[T1]{fontenc}
\usepackage{amsmath,amssymb, amsthm}

% \newcommand{\R}{\mathbb{R}}
% \newcommand{\N}{\mathbb{N}}

% \theoremstyle{plain}
% \newtheorem{teorema}{Teorema}

% \theoremstyle{plain}
% \newtheorem{proposizione}{Proposizione}

% \theoremstyle{definition}
% \newtheorem{definizione}{Definizione}

\author{Gianmarco Brocchi}
\title{Weighted Nonnegative Matrix Factorization
and Face Feature Extraction}
\subtitle{Un'implementazione}
\date{\today}

\begin{document}
\maketitle

Usando l'algoritmo suggerito nell'articolo di Vincent D. Blondel, Ngoc-Diep Ho e  Paul van Dooren si vuole applicare la WNMF ad una raccolta di immagini in modo da enfatizzare certe zone della matrice data. Verranno usate immagini di volti.

\section{L'algoritmo}
 
\section{I dati}
L'algoritmo lavora su dati ottenuti dall'ORL face database\footnote{un insieme di volti dall'Olivetti Research Laboratory di Cambridge, UK}.
Nello specifico si tratta di 400 volti (10 immagini di 40 soggetti). Le immagini sono in formato PGM\footnote{Portable Gray Map format}, ovvero in scala di grigi; tali immagini possono essere in formato RAW o ASCII, nel primo i valori di grigio sono memorizzati come semplici byte, nel secondo sono rappresentati da numeri tra 0 e 255. (Nota: le immagini dell'ORL erano nel primo formato, pertanto è stato necessario convertirle, vedi \ref{subsec:conversione})

\section{I test}
\subsection{Conversione}\label{subsec:conversione}

\end{document}

